\chapter[Medições Coletadas]{Medições Coletadas}\label{cap4}

Nessa sessão, será detalhada cada medição coledata nesse experimento, juntamente com o
objetivo.

\subsection{Medição 1}

A medição 1 é \textit{Estimar a o tamanho do desafio a ser implementado}. Consiste em pontuar, através do
scrum poker planner, o desafio proposto do ponto de vista do time do FF. Assim temos uma base para comparar a pespectiva do voluntário com
um valor base.

\subsection{Medição 2}

A medição 2 é \textit{Estimar a expectativa de tamanho da execução da atividade de criação de servidor com o framework}.
Consiste em pontuar, através do scrum poker planner, o desafio proposto do ponto de vista dos voluntários. Assim é possivel
comparar a grandeza do desafio do ponto de vista do voluntário para o time base do FF.

\subsection{Medição 3}

A medição 3 é \textit{Medir tempo da execução da atividade de criação de servidor com o framework}. Consistem em
coletar o tempo gasto por cada membro no desenvolvimento do desafio. Possibilitando o cálculo do custo como também
o da produtividade.

\subsection{Medição 4}

A medição 4 é \textit{Estimar a produtividade dos desenvolvedores na execução da atividade de criação de servidor com o framework.}. Consiste
em cálcular a divisão do tamanho extimado pelo tempo gasto. Possibilitando a comparação da eficiência entre cada abordagem
na resolução do desafio.

\subsection{Medição 5}

A medição 5 é \textit{Estimar a experiência em desenvolvimento de serviços web dos desenvolvedores a serem recrutados para teste}. Consite em
atribuir um nível de conhecimento para cada vonlutário para nivelamento dos sub-grupos de teste.

\subsection{Medição 6}

A medição 6 é \textit{Estimar dificuldade de uso do produto gerado.} Consiste em validar a usabilidade do FF. Uma vez que
mesmo sendo eficiente do ponto de vista técnico não é usual.

\subsection{Medição 7}

A mediçnao 7 é \textit{Estimar a satisfação de uso do produto gerado.} Consiste em avaliar o interesse do voluntáo no FF, após
a sua utilização.

\section{Indicadores}

Uma vez que todas as medições forem coletadas, é necessario a definição de indicadores baseados nessas medições.
Nesses trabalho foram definidos e cálculados quatro indicadores bases que são insumo para um indicador único que descreve
a operabilidade e a eficiencia do FF, conforme a imagem \ref{fig:indicadores}

\begin{figure}[h]
  \centering
  \label{fig:indicadores}
  \includegraphics[keepaspectratio=true,scale=0.8]{figuras/i.eps}
  \caption{Composição do indicador}
\end{figure}
