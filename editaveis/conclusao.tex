\chapter[Conclusão]{Conclusão}

Este trabalho teve o intuito de analisar a eficiência e operabilidade do Falcon Framework através
da realização de experimentos com grupos de desenvolvedores alvo e analise das medições
coletadas. O objetivo geral proposto para o trabalho foi alcançado, uma vez que todas
as medições e experimentos planejados foram execultados.

Através dos indicadores, podemos perceber que o FF teve exito em alcançar o seu objetivo, um vez que
a maioria dos indicadores - sessão \ref{ss:indicadores} - foram possitivos, isto é indicam que a utilização
do \textit{framework} trás melhorias no desenvolvimento de APIs.

Um fator interessante para se analisar é do indicador numero 2, presente na sessão \ref{ss:satifacao}.
O objetivo desse indicador é responder acerca da satisfação do usuário e teve-se resultado neutro, ou seja,
os desenvolvedores estão satisfeitos com FF assim como estão satisfeitos com o que já possuem conhecimento.
Apesar dos outros indicadores, onde vemos com clareza a produtividade e a eficiência que o FF trás, pode-se
dizer, pelo indicador 2, que alguns desenvolvedores não utilizariam o Falcon por estarem satisfeitos com
as técnologias atuais.

Sendo assim, conclui-se que o Falcon Framework funciona e que trás benefícios ao time de desenvolvimento, como
aumento na produtividade. Porém, a partir desse trabalho, percebeu-se que há a necessidade de investir
recursos no desenvolvemento do \textit{framework} sobre o ponto de vista da satisfação do usuário final.
