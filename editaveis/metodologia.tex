\chapter[Metodologia]{Metodologia}\label{cap3}

\subsection{Seleção dos Desenvolvedores}

Conforme previsto no plano de medição(anexo 1), a medição 5 tem como objetivo
Estimar a experiência em desenvolvimento de serviços web dos desenvolvedores
a serem recrutados para teste. Tornando assim, os dados coletados nivelados. Possibilitando
uma análise mais consistente.

Para a realização desse recrutamento e seguindo os critérios da medição de numero 5, foi
realizado um questionário[anexo 1] com o intuito de mapear os candidados do teste.

Nesse questionário, visa-se identificar os níveis de conhecimento sobre desenvolvimento
de APIs RESTFull como também a experiência no ambito de desenvolvimento em geral.

\subsection{Experimento Piloto}

Para a coleta das métricas necessárias para a avalição do FF, foi necessário a elaboração
de um desafio, conforme especificado no plano de medição(anexo 1). Para a elaboração, foi levado em consideração
fatores presentes na maioria das API desenvolvidas em projetos reais, exempo:

\begin{enumerate}
  \item Relacionamento de objetos;
  \item Auto relacionamento;
  \item Validação de dados;
\end{enumerate}

Uma vez que a maioria dos vonlutários estão familiarizados com sistemas acadêmicos, foi proposto
uma versão simplificada de um API para o gerenciamento de mátriculas, disciplicas e pré-requisitos.

\subsection{Execução do Experimento}

Para a execução do experimento, foram selecionados doze voluntários. Nivelados pelo grau de
conhecimento conforme descrito anteriormente em três sub-grupos.

Cada sub-grupo foi composto por quatro volutários com níveis diferentes em habilidade em api RESTFUL
  \begin{enumerate}
    \item Inexperiente
    \item Baixa Experiência
    \item Média Experiência
    \item Alta Experiência
  \end{enumerate}


Para todos os sub-grupos foi proposto o desenvolvimento do desafio elaborado conforme anteriormente dito. Contudo,
para um grupo não se apresentou o FF, deixando a escolha da plataforma de desenvolvimento livre. Para outro grupo,
foi apresentado o FF e deixando a escolha de cada um a plataforma para desenvolvimento. Por fim, o ultimo sub-grupo
foi apresentado o FF e solicitou-se que o desafio fosse desenvolvido utilizando-se o FF.

Cada sub-grupo teve 30 minutos para desenvolver o desafio, ao final do tempo todos pararam de desenvolver
e foi avaliado o desenpenho de cada um.
