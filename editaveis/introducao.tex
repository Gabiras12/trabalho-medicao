\chapter[Introdução]{Introdução}\label{cap1}

Para o desenvolvimento de uma API gasta-se uma grande quantidade de tempo na definição dos padrões de projeto que sejam utilizados. Como por exemplo: os padrões de requisição, padrões de resposta, padrões da distribuição e modularização do código. Após a definição de quais padrões eram utilizados há um necessidade de uma grande quantidade de tempo para a implementação e teste da API. Tornando, muitas vezes, o processo de desenvolvimento da API demorado e com custos elevados.

A proposta do Falcon Framework é justamente agilizar esse processo através da automatização da criação da API, utilizando-se dos padrões de projeto mais estáveis e de grande presença na comunidade.

Nesse trabalho, será detalhado a série de medições recolhidas em três experimentos
realizados com desenvolvedores.

Cada experimento foi realizado com uma grupo de quatro desenvolvedores e cada grupo nivelado
de acordo com o grau de conhecimento.
