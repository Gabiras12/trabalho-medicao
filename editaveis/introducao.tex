\chapter[Introdução]{Introdução}\label{cap1}

%O mercado referente aos áudiobooks tem crescido
%
%No capítulo \ref{cap2}, é apresentado os objetivos pretendidos para o qual este trabalho %foi motivado.%
%
%No capítulo \ref{cap3}, é apresentada toda a revisão bibliográfica onde foram revistos %monografias, dissertações, livros, publicações em \textit{websites} e especificações %necessários para o entendimento e desenvolvimento do projeto.%
%
%No capítulo \ref{cap4}, serão apresentados todas as etapas realizadas para a especificação %do formato e para o desenvolvimento do Editor bem como as ferramentas utilizadas com %suporte no processo de desenvolvimento e pesquisa.%
%
%No capítulo \ref{cap5}, serão apresentados os resultados obtidos no projeto, mostrando o %arquivo gerado pelo Editor.%
%
%No capítulo \ref{cap6}, 
%E, por fim, no capítulo \ref{cap7},

%Este documento apresenta considerações gerais e preliminares relacionadas 
%à redação de relatórios de Projeto de Graduação da Faculdade UnB Gama 
%(FGA). São abordados os diferentes aspectos sobre a estrutura do trabalho, 
%uso de programas de auxilio a edição, tiragem de cópias, encadernação, etc.

