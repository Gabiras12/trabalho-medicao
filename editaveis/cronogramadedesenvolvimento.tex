\chapter[Cronograma de Desenvolvimento]{Cronograma de Desenvolvimento}\label{cap7}

Para evidenciar o cronograma aproximado, segue o gráfico de \textit{gantt} representado pela Figura \ref{cronograma} mostrando o desenvolvimento do projeto que teve sua primeira semana iniciada no dia 16 de setembro de 2014. As entregas foram planejadas semanalmente. O gráfico mostra um aproximação pois algumas adaptações foram feitas no período de prototipagem.


 \begin{figure}[ht]
	\centering
		\includegraphics[keepaspectratio=true,scale=0.3]{figuras/cronograma.eps}
	\caption{Gráfico de \textit{gantt} referente ao cronograma.}
	\label{cronograma}
\end{figure}

\section{Trabalhos Futuros}

Neste trabalho houve a tentiva de criar um tocador para ser executado mesmo no terminal, mas não foi possível concluí-lo a tempo. Para continuação do trabalho, o próximo passo é implementar um player para o formato especificado em plataforma iOS. O aplicativo deverá ser capaz de executar o áudio, marcar conteúdo e salvá-lo no formato especificado e inserir metadados no mesmo. A plataforma iOS possui duas linguagens capazes de desenvolver os aplicativos: Objecitve-C e Swift. Será feita uma análise para verificar qual delas melhor encaixa no contexto do trabalho. 